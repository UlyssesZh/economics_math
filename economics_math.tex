\documentclass{article}
\usepackage{amsmath}
\usepackage{amsfonts}
\usepackage{amsthm}
\usepackage{hyperref}
\usepackage{mathtools}

\title{Calculus in microeconomics}
\author{Ulysses}

\newtheorem{axiom}{Axiom}
\newtheorem{theorem}{Theorem}
\newtheorem{corollary}{Corollary}
\newtheorem{definition}{Definition}

\begin{document}

\maketitle

\tableofcontents

\section{Free market of one good}

\subsection{Basic concepts and theorems}

\begin{definition}[quantity]
The \textbf{quantity} is a variable $Q\in\left[0,+\infty\right)$.
\end{definition}

The economic meaning of $Q$ is to measure how much good there is involved in the market within a unit of time.
Its unit is unit quantity per unit time.

\begin{definition}[price]
The \textbf{price} is a variable $P\in\left[0,+\infty\right)$.
\end{definition}

The economic meaning of $P$ is to measure the value of good.
Its unit is monetary unit per unit quantity.

\begin{definition}[cost and benefit]
The \textbf{cost} of the good is a function $C:\left[0,+\infty\right)\to\left[0,+\infty\right)$.

The \textbf{benefit} of the good is a function $B:\left[0,+\infty\right)\to\left[0,+\infty\right)$.
\end{definition}

$C$ maps $Q$ to the total cost $C\left(Q\right)$ (whose unit is monetary unit per unit time) for producers to produce such amount of good within a unit of time.
Similarly, $B$ maps $Q$ to the total benefit $B\left(Q\right)$ (whose unit is monetary unit per unit time) for consumers to consume such amount of good within a unit of time.

Here are some axioms concluded from principles of economics.

\begin{axiom}
\label{monoticity}
$C$ and $B$ are increasing on $\left[0,+\infty\right)$.
\end{axiom}

Axiom \ref{monoticity} is obvious when thinking about the economic meaning of $C$ and $B$.

\begin{axiom}
\label{smoothness}
$C$ and $B$ are twice differentiable on $\left[0,+\infty\right)$.
\end{axiom}

Axiom \ref{smoothness} is a mathematical requirement
without which we will lack mathematical tools to conveniently prove some theorems.

\begin{axiom}
\label{initial cost and benefit}
$$C\left(0\right)=B\left(0\right)=0.$$
\end{axiom}

Axiom \ref{initial cost and benefit} is as obvious as Axiom \ref{monoticity}.

\begin{axiom}
\label{initial supply and demand}
$$C'\left(0\right)<B'\left(0\right).$$
\end{axiom}

Axiom \ref{initial supply and demand} is sometimes optional,
but it is necessary if the market should exist.

\begin{axiom}
\label{inf}
$$\lim_{Q\to+\infty}C'\left(Q\right)=+\infty.$$
$$\lim_{Q\to+\infty}B'\left(Q\right)=0.$$
\end{axiom}

Axiom \ref{inf} is as necessary as Axiom \ref{initial supply and demand}.
Both of them restrict the boundary conditions of $C'$ and $B'$.

\begin{axiom}[law of supply and law of demand]
\label{law of supply and demand}
$C$ is strictly convex downward everywhere on $\left[0,+\infty\right)$.

$B$ is strictly concave downward everywhere on $\left[0,+\infty\right)$.
\end{axiom}

Axiom \ref{law of supply and demand} is a important and basic law in economics.
Note that law of demand is mathematically equivalent to law of diminishing marginal benefit.

From these axioms, we can easily deduce some corollaries using simple math.

\begin{theorem}
The inverse function of $C'$ and $B'$ exists if their codomains are restricted to their ranges.
\end{theorem}
\begin{proof}
According to Axiom \ref{law of supply and demand}, $C'$ is strictly increasing on $\left[0,+\infty\right)$.
According to Axiom \ref{inf}, $C'$ has range $\left[C'\left(0\right),+\infty\right)$.
Thus, we can define its inverse function $C'^{-1}:\left[C'\left(0\right),+\infty\right)\to\left[0,+\infty\right)$.

The proof for existence of $B'^{-1}$ is similar.
\end{proof}

\begin{definition}[supply and demand]
Function $C'$ is called \textbf{marginal cost (MC)} or \textbf{supply (S)}.

Function $B'$ is called \textbf{marginal benefit (MB)} or \textbf{demand (D)}.
\end{definition}

Marginal cost and marginal benefit shares the unit with $P$
(monetary unit per unit quantity).

It may be confusing why marginal cost is the same as supply.
Here is a simple explanation: producers do not operate at a price below the shutdown point.
It is similar to explain why marginal benefit is the same as demand.

By convention in the context of supply and demand graphs, the inverse functions $C'^{-1}$ and $B'^{-1}$ are used instead of $C'$ and $B'$.

\begin{definition}[surplus]
\label{surplus}
The \textbf{producer surplus} $S_C:\left[0,+\infty\right)^2\to\mathbb R$ is defined as
$$S_C\left(Q,P\right):=PQ-C\left(Q\right).$$

The \textbf{consumer surplus} $S_B:\left[0,+\infty\right)^2\to\mathbb R$ is defined as
$$S_B\left(Q,P\right):=B\left(Q\right)-PQ.$$

The \textbf{social surplus} $S:\left[0,+\infty\right)\to\mathbb R$ is defined as
$$S:=B-C.$$
\end{definition}

The unit of surplus is monetary unit per unit time.
The economic meaning of surplus is the net gain within a unit of time.
For producers, the net gain is part of income $PQ$ that exceeds the cost $C\left(Q\right)$.
For consumers, the net gain is part of benefit $B\left(Q\right)$ that exceeds the expenditure $PQ$.

\begin{theorem}
For any $Q\ge0$,
$$S_C\left(Q,C'\left(Q\right)\right)\ge0,$$
$$S_B\left(Q,B'\left(Q\right)\right)\ge0,$$
where the equality holds only when $Q=0$.
\end{theorem}
\begin{proof}
\begin{align*}
S_C\left(Q,C'\left(Q\right)\right)&=C'\left(Q\right)Q-C\left(Q\right)\qquad\text{(Definition \ref{surplus})}\\
&=\int_0^QC'\left(Q\right)\mathrm dx-\int_0^QC'\left(x\right)\mathrm dx\\
&=\int_0^Q\left(C'\left(Q\right)-C'\left(x\right)\right)\mathrm dx.
\end{align*}
According to Axiom \ref{law of supply and demand}, $C'$ is strictly increasing,
so when $Q>0$, for any $x\in\left[0,Q\right)$, $C'\left(Q\right)-C'\left(x\right)>0$.
Thus, the integral is an integral of an always-positive function, which means
$$S_C\left(Q,C'\left(Q\right)\right)>0.$$

When $Q=0$, it can be easily proved that $S_C\left(Q,C'\left(Q\right)\right)=0$.

It is similar to show that $$S_B\left(Q,B'\left(Q\right)\right)>0.$$
\end{proof}

\begin{theorem}
$Q\mapsto S_C\left(Q,C'\left(Q\right)\right)$ is strictly increasing.

$Q\mapsto S_B\left(Q,B'\left(Q\right)\right)$ is strictly increasing.
\end{theorem}
\begin{proof}
\begin{align*}
\frac{\mathrm d}{\mathrm dQ}S_C\left(Q,C'\left(Q\right)\right)&=\frac{\mathrm d}{\mathrm dQ}\left(C'\left(Q\right)Q-C\left(Q\right)\right)\qquad\text{(Definition \ref{surplus})}\\
&=C''\left(Q\right)Q\\
&>0\qquad\text{(Axiom \ref{law of supply and demand})},
\end{align*}
so $Q\mapsto S_C\left(Q,C'\left(Q\right)\right)$ is strictly increasing.

It is similar to prove that $Q\mapsto S_B\left(Q,B'\left(Q\right)\right)$ is strictly increasing.
\end{proof}

\begin{theorem}
For any $Q,P\in\left[0,+\infty\right)$,
$$S\left(Q\right)=S_C\left(Q,P\right)+S_B\left(Q,P\right),$$
which is independent of $P$.
\end{theorem}
\begin{proof}
\begin{align*}
S\left(Q\right)&=B\left(Q\right)-C\left(Q\right)\qquad\text{(Definition \ref{surplus})}\\
&=B\left(Q\right)-PQ+PQ-C\left(Q\right)\\
&=S_C\left(Q,P\right)+S_B\left(Q,P\right)\qquad\text{(Definition \ref{surplus})}.
\end{align*}
\end{proof}

\begin{definition}[price elasticity]
\label{price elasticity}
The \textbf{price elasticity} of a function $f:\left[0,+\infty\right)\to\left[0,+\infty\right)$ with non-zero derivative everywhere on $\left(0,+\infty\right)$
is a function $E_f:\left(0,+\infty\right)\to\mathbb R$ defined as
$$E_f\left(Q\right):=\frac{f\left(Q\right)}{f'\left(Q\right)Q}.$$
\end{definition}

In Definition \ref{price elasticity}, the involved function $f$ is a function mapping $Q$ to $P$, usually $C'$ or $B'$.
In these two special cases, the price elasticity is called \textbf{price elasticity of supply (PES)} or \textbf{price elasticity of demand (PED)}.
The economic meaning of price elasticity is to show the responsiveness (elasticity) of $Q$ to $P$.
It is the ratio of percentage change in $Q$ and that in $P$.

\begin{theorem}
\label{sign of price elasticity}
For any $Q>0$,
$$E_{B'}\left(Q\right)<0,$$
$$E_{C'}\left(Q\right)>0.$$
\end{theorem}
\begin{proof}
According to Definition \ref{price elasticity},
$$E_{B'}\left(Q\right)=\frac{B'\left(Q\right)}{B''\left(Q\right)Q}.$$
According to Axiom \ref{monoticity}, $B'\left(Q\right)>0$.
According to Axiom \ref{law of supply and demand}, $B''\left(Q\right)<0$.
Thus, $E_{B'}\left(Q\right)<0$.
\end{proof}

\begin{theorem}
\label{upper bound of benefit}
If $\lim_{Q\to+\infty}E_{B'}\left(Q\right)\in\left(-1,0\right]$, then $B$ has upper bound.
\end{theorem}
\begin{proof}
According to Theorem \ref{sign of price elasticity},
$\lim_{Q\to+\infty}E_{B'}\left(Q\right)$ cannot be positive.

Case 1: $\lim_{Q\to+\infty}E_{B'}\left(Q\right)=0$.

According to definition of limit, for any $\varepsilon>0$, there exists $n$ such that for any $Q>n$, we have
$$\left|E_{B'}\left(Q\right)\right|<\varepsilon.$$
According to Theroem \ref{sign of price elasticity}, $E_{B'}\left(Q\right)<0$.
Thus, the equation can be written as
$$E_{B'}\left(Q\right)>-\varepsilon.$$
Substitute Definition \ref{price elasticity}, and let $m:=\frac1\varepsilon$, and then we can derive
$$\frac{B'\left(Q\right)}{B''\left(Q\right)Q}>-\frac1m,$$
which means
$$\frac{B''\left(Q\right)}{B'\left(Q\right)}<-\frac mQ.$$
Note that it can be written as
$$\frac{\mathrm d}{\mathrm dQ}\ln\left(B'\left(Q\right)\right)<-\frac{\mathrm d}{\mathrm dQ}\left(m\ln Q\right),$$
or
$$\frac{\mathrm d}{\mathrm dQ}\ln\left(B'\left(Q\right)Q^m\right)<0.$$
This means $B'\left(Q\right)Q^m$ is strictly decreasing on $\left(n,+\infty\right)$,
so for any $Q>q>n$,
$$B'\left(Q\right)Q^m<B'\left(q\right)q^m,$$
which means that for any $Q>q$,
$$B'\left(Q\right)<CQ^{-m},$$
where $C:=B'\left(q\right)q^m$.

Write $\lim_{Q\to+\infty}B\left(Q\right)$ in the integral form
\begin{align*}
\lim_{Q\to+\infty}B\left(Q\right)&=B\left(q\right)+\int_q^{+\infty}B'\left(Q\right)\mathrm dQ\\
&<B\left(q\right)+\int_q^{+\infty}CQ^{-m}\mathrm dQ.
\end{align*}
Since $m$ is an arbitrary positive number, we can take $m>1$.
In this way, the integral above converges, and we can derive that
$$\lim_{Q\to+\infty}B\left(Q\right)<B\left(q\right)+\frac C{m-1}q^{1-m}.$$
Thus, according to comparison test, $\lim_{Q\to+\infty}B\left(Q\right)$ must exist.

It can be easily proved that $\lim_{Q\to+\infty}B\left(Q\right)$ is the supremum of $B$, so $B$ has upper bound.

Case 2: $\lim_{Q\to+\infty}E_{B'}\left(Q\right)\in\left(-1,0\right)$.

Let
$$l:=-\frac1{\lim_{Q\to+\infty}E_{B'}\left(Q\right)}\in\left(1,+\infty\right),$$
and then we have
$$\lim_{Q\to+\infty}\frac1{E_{B'}\left(Q\right)}=-l.$$

According to definition of limit, for any $m>0$, there exists $n$ such that for any $Q>n$, we have
$$\left|\frac1{E_{B'}\left(Q\right)}+l\right|<m,$$
so
$$\frac1{E_{B'}\left(Q\right)}<m-l.$$
Substitute Definition \ref{price elasticity}, and then we can derive
$$\frac{B''\left(Q\right)Q}{B'\left(Q\right)}<m-l.$$
Use similar tricks in Case 1, and then we can derive
$$\frac{\mathrm d}{\mathrm dQ}\ln\left(B'\left(Q\right)Q^{l-m}\right)<0.$$
Use similar tricks in Case 1, and take $m\in\left(0,l-1\right)$,
and we can prove that $B$ has upper bound.
\end{proof}

\begin{theorem}
If $B$ has upper bound, then $\lim_{Q\to+\infty}E_{B'}\left(Q\right)$ can be any value in $\left[-1,0\right]$, and can also non-exist, and cannot be less than $-1$.
\end{theorem}
\begin{proof}
Example where $\lim_{Q\to+\infty}E_{B'}\left(Q\right)$ does not exist:

Let
$$B\left(Q\right):=\int_0^Q\frac{2+\sin\ln\left(1+x\right)}{\left(1+x\right)^2}\mathrm dx.$$
Then, $B$ satisfies Axiom \ref{monoticity}, Axiom \ref{smoothness}, Axiom \ref{initial cost and benefit}, Axiom \ref{inf}, and Axiom \ref{law of supply and demand} as can be justified.

$B$ has upper bound because
\begin{align*}
B\left(Q\right)&<\int_0^Q\frac3{\left(1+x\right)^2}\mathrm dx\\
&<3\int_1^{+\infty}\frac{\mathrm dx}{x^2}\\
&=3.
\end{align*}

Calculate $E_{B'}\left(Q\right)$, and we have
\begin{align*}
E_{B'}\left(Q\right)&:=\frac{B'\left(Q\right)}{B''\left(Q\right)Q}\qquad\text{(Definition \ref{price elasticity})}\\
&=\frac{1+x}x\cdot\frac{2+\sin\ln\left(1+x\right)}{-2\left(2+\sin\ln\left(1+x\right)\right)+\cos\ln\left(1+x\right)}.
\end{align*}
When $x=\mathrm e^{2k\pi}-1$,
$$E_{B'}\left(Q\right)=\frac{1+x}x\cdot\frac{-2}3\xrightarrow{x\to+\infty}-\frac23.$$
When $x=\mathrm e^{\left(k+\frac14\right)2\pi}-1$,
$$E_{B'}\left(Q\right)=\frac{1+x}x\cdot\frac{-1}2\xrightarrow{x\to+\infty}-\frac12.$$
($k\in\mathbb N$.)

Thus, $\lim_{Q\to+\infty}E_{B'}\left(Q\right)$ does not exist.

Example where $\lim_{Q\to+\infty}E_{B'}\left(Q\right)=0$:

Let
$$B\left(Q\right):=1-\mathrm e^{-Q}.$$
Then, $B$ satisfies Axiom \ref{monoticity}, Axiom \ref{smoothness}, Axiom \ref{initial cost and benefit}, Axiom \ref{inf}, and Axiom \ref{law of supply and demand}, and has upper bound $1$ as can be justified.

Calculate $E_{B'}\left(Q\right)$, and we have
\begin{align*}
E_{B'}\left(Q\right)&:=\frac{B'\left(Q\right)}{B''\left(Q\right)Q}\qquad\text{(Definition \ref{price elasticity})}\\
&=-\frac1Q\\
&\xrightarrow{Q\to+\infty}0.
\end{align*}

Example where $\lim_{Q\to+\infty}E_{B'}\left(Q\right)=-1$:

Let
$$B\left(Q\right):=1-\frac1{\ln\left(\mathrm e+Q\right)}.$$
Then, $B$ satisfies Axiom \ref{monoticity}, Axiom \ref{smoothness}, Axiom \ref{initial cost and benefit}, Axiom \ref{inf}, and Axiom \ref{law of supply and demand}, and has upper bound $1$ as can be justified.

Calculate $E_{B'}\left(Q\right)$, and we have
\begin{align*}
E_{B'}\left(Q\right)&:=\frac{B'\left(Q\right)}{B''\left(Q\right)Q}\qquad\text{(Definition \ref{price elasticity})}\\
&=-\frac{\mathrm e+Q}Q\cdot\frac{\ln\left(\mathrm e+Q\right)}{2+\ln\left(\mathrm e+Q\right)}\\
&\xrightarrow{Q\to+\infty}-1.
\end{align*}

Example where $\lim_{Q\to+\infty}E_{B'}\left(Q\right)$ can be any value in $\left(-1,0\right)$:

Let
$$B\left(Q\right):=1-\left(1+x\right)^{1+\frac1l},$$
where $l\in\left(-1,0\right)$ is an arbitrary parameter.
Then, $B$ satisfies Axiom \ref{monoticity}, Axiom \ref{smoothness}, Axiom \ref{initial cost and benefit}, Axiom \ref{inf}, and Axiom \ref{law of supply and demand}, and has upper bound $1$ as can be justified.

Calculate $E_{B'}\left(Q\right)$, and we have
\begin{align*}
E_{B'}\left(Q\right)&:=\frac{B'\left(Q\right)}{B''\left(Q\right)Q}\qquad\text{(Definition \ref{price elasticity})}\\
&=l\left(1+\frac1Q\right)\\
&\xrightarrow{Q\to+\infty}l.
\end{align*}

Prove that $\lim_{Q\to+\infty}E_{B'}\left(Q\right)\ge-1$:

Use means of contradiction.

Suppose $\lim_{Q\to+\infty}E_{B'}\left(Q\right)<-1$,
and use similar tricks in the proof of Theorem \ref{upper bound of benefit}.
We can derive that there exists $q>0$ such that for any $Q>q$,
$$B'\left(Q\right)Q>B'\left(q\right)q,$$
which means
$$B'\left(Q\right)>\frac Cx,$$
where $C:=B'\left(q\right)q$.

Integrate both sides, and we can derive
$$B\left(Q\right)>B\left(q\right)+C\ln\frac Qq\xrightarrow{Q\to+\infty}+\infty,$$
which contradicts with it that $B$ has upper bound.
\end{proof}

\begin{definition}[revenue]
\label{revenue}
The \textbf{revenue} of a function $f:\left[0,+\infty\right)\to\left[0,+\infty\right)$ is a function $R_f:\left[0,+\infty\right)\to\left[0,+\infty\right)$ defined as
$$R_f\left(Q\right):=f\left(Q\right)Q.$$
\end{definition}

In Definition \ref{revenue}, the involved function $f$ is a function mapping $Q$ to $P$, usually $C'$ or $B'$.
One may usually find that the surplus can often be represented with difference of revenue and cost/benefit.

\begin{theorem}
\label{price elasticity and revenue}
If $f:\left[0,+\infty\right)\to\left[0,+\infty\right)$ is a function with non-zero derivative everywhere on $\left(0,+\infty\right)$, and $Q_0\ne0$ is an extremal of $R_f$ such that $f\left(Q_0\right)\ne0$, then $E_f\left(Q_0\right)=-1$.
\end{theorem}
\begin{proof}
Because $R_f$ is a differentiable function, its extremal is a solution of equation $R_f'=0$ or the boundary of its domain.
Since $Q_0\ne0$, $Q_0$ is{} not the boundary of the domain of $R_f$.
Thus $R_f'\left(Q_0\right)=0$.
\begin{align*}
R_f'\left(Q\right)&=\frac{\mathrm d}{\mathrm dQ}\left(f\left(Q\right)Q\right)\qquad\text{(Definition \ref{revenue})}\\
&=f'\left(Q\right)Q+f\left(Q\right)\\
&=\frac{f\left(Q\right)}{E_f\left(Q\right)}+f\left(Q\right)\qquad\text{(Definition \ref{price elasticity})}\\
&=\left(1+\frac1{E_f\left(Q\right)}\right)f\left(Q\right).
\end{align*}
From the calculations above, since $R_f'\left(Q_0\right)=0$ and $f\left(Q_0\right)\ne0$, we have $E_f\left(Q\right)=-1$.
\end{proof}

\begin{theorem}
\label{inequality of revenue and supply and demand}
For any $Q\ge0$,
$$C'\left(Q\right)\le R_{C'}'\left(Q\right),$$
$$B'\left(Q\right)\ge R_{B'}'\left(Q\right),$$
where the equality holds only when $Q=0$.
\end{theorem}
\begin{proof}
Consider $Q>0$, and then we have
\begin{align*}
R_{C'}'\left(Q\right)&=\left(1+\frac1{E_{C'}\left(Q\right)}\right)C'\left(Q\right)\qquad\text{(Theorem \ref{price elasticity and revenue})}\\
&>C'\left(Q\right)\qquad\text{(Theorem \ref{sign of price elasticity})}.
\end{align*}
As for $Q=0$, it can be easily proved that $R_{C'}'\left(0\right)=C'\left(0\right)$.

It is similar to prove $B'\left(Q\right)\ge R_{B'}'\left(Q\right)$.
\end{proof}

\begin{theorem}
If $B$ has upper bound, then $\lim_{Q\to+\infty}R_{B'}\left(Q\right)=0$.
\end{theorem}
\begin{proof}
Because $B$ has upper bound and is strictly increasing, $\lim_{Q\to+\infty}B\left(Q\right)$ exists.
According to Cauchy's convergence test, for any $\frac\varepsilon2>0$, there exists $n$ such that
when $Q>\frac Q2>n$,
$$B\left(Q\right)-B\left(\frac Q2\right)<\frac\varepsilon2$$
(the absolute value bracket is omitted due to Axiom \ref{monoticity}).

On the other hand,
\begin{align*}
B\left(Q\right)-B\left(\frac Q2\right)&=\int_{\frac Q2}^QB'\left(x\right)\mathrm dx\\
&>\int_{\frac Q2}^QB'\left(Q\right)\mathrm dx\qquad\text{(Axiom \ref{law of supply and demand})}\\
&=\frac12B'\left(Q\right)Q\\
&=\frac12R_{B'}\left(Q\right)\qquad\text{(Definition \ref{revenue})}.
\end{align*}

Thus, when $Q>n$, $0<R_{B'}\left(Q\right)<\varepsilon$, which means $\lim_{Q\to+\infty}R_{B'}\left(Q\right)=0$.
\end{proof}

\subsection{Perfect competition}

\begin{theorem}
\label{perfect competition equilibrium}
$S$ has a unique maximal on $\left(0,+\infty\right)$,
which is the unique solution of equation $C'=B'$.
\end{theorem}
\begin{proof}
It can be just proved by showing that the equation $C'=B'$ has a unique solution on $\left(0,+\infty\right)$.
In other words, if we construct a function $$f:=S'=C'-B',$$ then $f$ has a unique zero on $\left(0,+\infty\right)$.

According to Axiom \ref{initial supply and demand}, we have
$$f\left(0\right)=C'\left(0\right)-B'\left(0\right)<0.$$

According to Axiom \ref{law of supply and demand}, $C'$ is strictly increasing, and $B'$ is strictly decreasing, so $f$ is strictly increasing.

According to Axiom \ref{inf}, we have
$$\lim_{x\to\infty}f\left(x\right)=+\infty,$$
which means that $f$ has range $\left[f\left(0\right),+\infty\right)$,
so $f$ has its inverse function $f^{-1}:\left[f\left(0\right),+\infty\right)\to\left[0,+\infty\right)$.

Because $f\left(0\right)<0$, we have $0\in\left[f\left(0\right),+\infty\right)$.
Thus, $f^{-1}\left(0\right)$ is the unique solution of equation $f=0$.

After that, it can be easily proved that $f^{-1}\left(0\right)\in\left(0,+\infty\right)$ is the unique maximal of $S$.
\end{proof}

According to principles of economics,
the maximal of $S$ mentioned in Theorem \ref{perfect competition equilibrium}
is the \textbf{equilibrium of a perfect competition market}.

\subsection{Monopoly and monopsony}

\begin{theorem}
\label{monopoly and monopsony equilibrium}
$Q\mapsto S_C\left(Q,B'\left(Q\right)\right)$ has at least one maximal on $\left(0,+\infty\right)$, which is a solution of equation $R_{B'}'=C'$.

$Q\mapsto S_B\left(Q,C'\left(Q\right)\right)$ has at least one maximal on $\left(0,+\infty\right)$, which is a solution of equation $R_{C'}'=B'$.
\end{theorem}
\begin{proof}
According to Definition \ref{surplus} and Definition \ref{revenue},
$$f\left(Q\right):=S_C\left(Q,B'\left(Q\right)\right)=R_{B'}\left(Q\right)-C\left(Q\right).$$

According to Axiom \ref{initial supply and demand} and Theorem \ref{inequality of revenue and supply and demand},
$$f'\left(0\right)=B'\left(0\right)-C'\left(0\right)>0,$$
so $0$ cannot be a maximal of $f$.

According to Axiom \ref{inf} and Theorem \ref{inequality of revenue and supply and demand},
$$\lim_{Q\to+\infty}f'\left(Q\right)<\lim_{Q\to+\infty}\left(B'\left(Q\right)-C'\left(Q\right)\right)=-\infty.$$
Thus, using definition of limit, it can be easily proved that there exists $Q_0>0$ such that $f'\left(Q_0\right)<0$,
so $Q_0$ cannot be a maximal of $f$.

Since $f$ is continuous, its range on closed interval $\left[0,Q_0\right]$ is a closed interval.
Suppose the supremum of the range is $f\left(z\right)$, where $z\in\left(0,Q_0\right)$.
Then $z\in\left(0,+\infty\right)$ is a maximal of $f$.

Because $f$ is differentiable everywhere, its maximal must be a solution of equation $f'=0$,
which is the same equation as $R_{B'}'=C'$.

It is similar to prove that $Q\mapsto S_B\left(Q,C'\left(Q\right)\right)$ has at least one maximal on $\left(0,+\infty\right)$, which is a solution of equation $R_{C'}'=B'$.
\end{proof}

According to principles of economics,
the maximal of $Q\mapsto S_C\left(Q,B'\left(Q\right)\right)$ mentioned in Theorem \ref{monopoly and monopsony equilibrium}
is the \textbf{equilibrium of a monopoly market}.
Similarly, the maximal of $Q\mapsto S_B\left(Q,C'\left(Q\right)\right)$ mentioned in \ref{monopoly and monopsony equilibrium}
is the \textbf{equilibrium of a monopsony market}.

\section{Market of one good with gov intervention}

\subsection{Taxes and subsidies}

\subsection{Price controls}

\section{Free market of multiple goods}

\subsection{Basic concepts}

\end{document}
